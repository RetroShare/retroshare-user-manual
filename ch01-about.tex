\inGerman{\section{Über dieses Dokument}}\inEnglish{\section{About this document}}

\inGerman{Dieses Dokument ist eine inoffizielle Anleitung für das Computer-Programm \href{http://retroshare.sourceforge.net/}{RetroShare}.}
\inEnglish{This document is an inofficial user manual for the program \href{http://retroshare.sourceforge.net/}{RetroShare}.}

\inGerman{Ich habe dieses Handbuch in \LaTeX geschrieben, weil ich RetroShare für ein geniales Projekt mit großem Potenzial halte und es unterstützen möchte.
Da ich keine Zeit habe, mich in die Details des Quellcodes einzuarbeiten, möchte ich meinen Beitrag zu RetroShare hiermit leisten.
Sämtliches Wissen über RetroShare habe ich mir durch Ausprobieren, lesen von RetroShare-internen Foren und gelegentliches in den Quellcode schauen angeeignet.
Ich bin insbesondere kein Entwickler von RetroShare und es ist möglich, dass hier einige Details falsch sind.}
\inEnglish{I wrote this manual in \LaTeX, because I'm a big fan of RetroShare and want to support it. As coding would take too much of my time, I wrote this manual instead. I got my knowledge about RetroShare mainly via try and error, reading the forums and looking in the code sometimes.
I'm NOT a developer of RetroShare and it some minor details here might be wrong.}

\inGerman{Wenn du also in diesem Dokument inhaltliche Fehler gefunden hast, wäre ich um eine kurze Rückmeldung sehr dankbar.
Auch wenn du mithelfen möchtest, dieses Dokument zu verbessern (z.B. vervollständigen, Übersetzung in andere Sprachen, schöneres Deutsch xD ...) und zumindest die Grundlagen von \LaTeX kannst, so nehme bitte mit mir Kontakt auf.
Meine Kontaktdaten finden sich auf meinem privaten Blog unter der Adresse \url{http://yet-another-nerd-blog.de}.}
\inEnglish{If you noticed an error in this document, I'd appreciate a short feedback.
If you wanna help improving this document and you know at least the basics of \LaTeX, so contact me.
My contact information can be found on my private blog at \url{http://yet-another-nerd-blog.de}.}
