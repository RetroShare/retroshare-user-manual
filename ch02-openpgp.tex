\section{\inGerman{Eine kurze Einführung in OpenPGP}\inEnglish{A short introduction to OpenPGP}}

\inGerman{Da RetroShare stark auf OpenPGP und dessen Funktionsweise aufbaut, soll hier eine (sehr kurze) Einführung dazu stattfinden.
Wer sich damit schon auskennt, kann ohne weiteres zum nächsten Kapitel springen.
Über GnuPG und Email-Verschlüsselung findet man im Internet schon Dokumente zuhauf, wir wollen hier nur auf die absoluten Grundlagen eingehen.}
\inEnglish{Since RetroShare makes heavy use of OpenPGP, we want to make a (very) short introduction to asymmetric encryption here.
If you know about OpenPGP already, you can proceed to the next chapter.}

\subsection{\inGerman{asymmetrische Verschlüsselung}\inEnglish{asymmetric encryption}}
\inGerman{OpenPGP ermöglicht eine asymmetrische Verschlüsselung.}
\inEnglish{OpenPGP uses asymmetric encryption.}

\inGerman{Dies bedeutet, dass jeder Teilnehmer sich sowohl einen \emph{öffentlichen} und einen dazugehörigen \emph{privaten} Schlüssel generiert.
Der öffentliche Schlüssel wird an alle deine Freunde verteilt und ermöglicht die Verschlüsselung von geheimen Nachrichten an dich.
Einmal mit dem öffentlichen Schlüssel verschlüsselt, könnte diese verschlüsselte Nachricht nach heutigem Wissenstand mit sämtlichen Rechnern der Welt nicht in den nächsten 100 Jahren entschlüsselt werden.
Nur mit Kenntnis des privaten Schlüssels ist dies möglich, dieser befindet sich jedoch nur auf deiner Festplatte und wird niemals an dritte weitergegeben.}
\inEnglish{This means, that every participant creates a \emph{public} and a corresponding \emph{private} key.
The public key is spread to all friends and allows them to encrypt messages for you.
If a message is encrypted with a public key, only persons with the private key can decrypt this message.
But the only person, who has the private key belonging to your public key, is you and so only you can read the message. 
This is the idea behind asymmetric encryption.}

\inGerman{Man kann mit asymmetrischer Verschlüsselung auch die Echtheit von Nachrichten sicherstellen, indem der Absender diese \emph{signiert}.
Das Signieren einer Nachricht, vergleichbar mit einer Unterschrift unter einem Dokument, kann dabei nur mit dem privaten Schlüssel des Schlüssels stattfinden.
Die Überprüfung ob eine Signatur in Ordnung ist, kann jeder mit dem öffentlichen Schlüssel durchführen.}
\inEnglish{You can use asymmetric Encryption to ensure the authenticity of messages, which is called \emph{signing}.
In fact, you can compare it to a signature in real life, as only you with your private key are able to create it.
Everyone, who owns the public key, can check the signature then.}


\subsection{Web of Trust}
\inGerman{Ein grundlegendes Problem der \emph{asymmetrischen Verschlüsselung} ist der initiale Schlüsselaustausch.
Möchten zwei Personen (Alice und Bob) miteinander sicher kommunizieren, muss zunächst jeder den öffentlichen Schlüssel des anderen kennen.
Wird dieser Austausch nun "uber das Internet erledigt, so könnte nun eine dritte Person diesen manipulieren und sich somit \emph{zwischenschalten}, ein sog. \emph{Man-in-the-Middle}-Angriff.}
\inEnglish{A basic problem is the initial exchange of keys between two friends.
If Alice and Bob want to use asymmetric encryption, they will have to know each other's public key first.
A malicious third party could intercept this exchange - a so called ``Man-in-the-middle attack''.}

\inGerman{Um dies zu verhindern, bietet GnuPG die Möglichkeit, die Echtheit des ausgetauschten Schlüssels mittels Fingerabdruck zu verifizieren.
Dabei sollte dieser Fingerabdruck über ein zweites und sicheres System (SMS, Telefon, Keysigning-Party oder Post) ausgetauscht werden.
Wenn man sich davon überzeugt hat, das Schlüssel und Person zusammengehören, \emph{\textbf{kann}} man das Anderen mittels unterschreiben des Schlüssels mitteilen. 
Damit wird das sogenannte \emph{Web of Trust} aufgebaut.}
\inEnglish{To prevent such attacks, GnuPG allows persons to \emph{sign keys}.
If you transferred the key manually, or you checked it via a safe channel like telephone, you should sign your friends key.
The more signatures a key has, the more you can be sure, that it's the actual key and not a key created by an attacker.
This whole process of signing other keys is called the \emph{Web of Trust}.}

\inGerman{Für RetroShare sind Unterschriften auf Schlüsseln und die Einstellung \emph{Vertrauen} nicht von Bedeutung, es werden alle Freunde gleich behandelt.}
\inEnglish{RetroShare doesn't differ between signed keys and not signed keys, all friends are treated equal.}

\inGerman{Das war schon alles, was wir wissen m\"ussen, wer mehr wissen will, dem seien die Wikipedia Artikel \emph{Asymetrische Verschlüsselung} und \emph{Web of Trust} nahegelegt.}
\inEnglish{That's all we need to know for using RetroShare. If you are interested in details, I'll suggest reading the wikipedia articles \href{http://en.wikipedia.org/wiki/Public-key_cryptography}{Public-Key Cryptography} and \href{http://en.wikipedia.org/wiki/Web_of_trust}{Web of Trust}}.
